\needspace{6em}
{\Large {\bf Professional Experience}} 
\vspace{0.1cm}
\hrule
\medskip

\needspace{6em}
\begin{tabular*}{7.1in}{@{}l@{\extracolsep\fill}r}
{\large {\bf Swarthmore College}} & Swarthmore, PA, Visiting Assistant Professor, 2016 - Current\\
\end{tabular*}

As a faculty member at Swarthmore College, my responsibilities include
teaching, mentoring, and research. In mentoring, she has worked with students to create 
their own 3D applications, including fluid simulators, (tile warp thing), and digital cities. 

Since then, her research has primarily looked at how the
animation of avatars -- e.g. how we represent ourselves digitally -- affects
how users perceive themselves.

Aline really enjoys collaborating with others to help them use computer
graphics in their own work. She is currently working with artists to build a
collaborative drawing application based on rasberry pis; with designers to
preview monuments in public spaces using an augmented reality; and with sign
language experts to analyze hand motion capture data. If you've been curious
about how to create your own avatars, 3D games, or VR/AR experiences, please
feel free to reach out. She might be able to point you towards resources for
getting started. 

\medskip
\medskip

\needspace{6em}
\begin{tabular*}{7.1in}{@{}l@{\extracolsep\fill}r}
{\large {\bf Savvy Sine, LLC}} & Philadelphia, PA, Director of Software Development, 2018 - Current\\
\end{tabular*}

motion capture analysis consulting, a statistics
teaching tool built in Unity, and a virtual reality escape room game built in

\needspace{6em}
\begin{tabular*}{7.1in}{@{}l@{\extracolsep\fill}r}
{\large {\bf Venturi Labs, LLC}} & Philadelphia, PA, Director of Software Development, 2017 - Current\\
\end{tabular*}

I work with students and industry partners to design and implement cutting edge
games and trainers for PC, phone, and virtual reality. Company projects include
a drone racing game built on Cesium, a virtual reality escape room game, and an
application  to preview monuments in public spaces using an augmented reality.

\medskip
\medskip

\needspace{6em}
\begin{tabular*}{7.1in}{@{}l@{\extracolsep\fill}r}
{\large {\bf Moon Collider, Ltd}} & Edinburgh, UK, AI Programmer and Researcher, 2015-2016\\
\end{tabular*}

Researched and implemented navigation, steering, and decision-making algorithms for Kythera, Moon Collider's highly 
scalable, game AI middleware. Implemented ORCA-based 2D avoidance algorithms used in the indie hack-n-slash \emph{Wolcen: Lords of Mayhem} and the survival sandbox zombie game, \emph{Miscreated}. Scripted NPCs with behavior trees for \emph{Aquanox: Deep Descent}.

\medskip
\medskip

\needspace{6em}
\begin{tabular*}{7.1in}{@{}l@{\extracolsep\fill}r}
{\large {\bf SIG Center, University of Pennsylvania}} & Philadelphia, PA, Associate Director, 2012-2013\\
\end{tabular*}

Managed the computer graphics facilities for all PhD, CGGT, and DMD students (100+ students). Maintained and managed the motion capture facilities and supervised capture sessions and data cleanup. Mentored 20+ undergraduate projects, ran workshops and lab tours for K-12 outreach in science and engineering. Setup, maintained, and provided training for various lab software and hardware, such as our VICON motion capture system (with forceplate, sole pressure sensors and eye tracker integration), renderfarm, backup storage systems, SVN repository, graphics cards, and development environments (openCV, QT, Unity, Maya, Nexus, etc)

\medskip

\needspace{6em}
\begin{tabular*}{7.1in}{@{}l@{\extracolsep\fill}r}
{\large {\bf ACASA, University of Pennsylvania}} & Philadelphia, PA, Sr. Programmer/Analyst, 2006-2008\\
\end{tabular*}

{\bf NonKin Village}\\
Managed, designed, and developed the simulation of a fictional Iraqi town (100+ residents) for a counter insurgency training game; built narrative authoring tools and interactive 2D map framework; wrote budget and scheduling plans, technical proposals; and presented projects to contract sponsors.

{\bf PMFServ}\\
Overhauled software framework for designing and testing agents: Improved performance 90+\%; Designed new ``Agile Agent Architecture" and implemented new plugin framework; Streamlined GUI; Ported social models to framework.

{\bf InsurgiSim and CrowdSim}\\
Managed and mentored student teams for insurgent simulation projects; Integrated student work with JSAF and spearheaded the final, stable executables. 

\medskip
\medskip

\needspace{6em}
\begin{tabular*}{7.1in}{@{}l@{\extracolsep\fill}r}
{\large {\bf MAK Technologies}} & Cambridge, MA, Senior Software Engineer, 1999-2006\\
\end{tabular*}

Senior developer for applications, utilities, toolkits, and demos for distributed simulation tools.  Developed over five new applications and APIs (contributing to six out of ten of MAK's 2006 product suite); developed internal applications for license management and demo creation; worked daily to solve customer problems; wrote documentation and training materials; demoed software at over 15 tradeshows and workshops, and released over 30 product distributions. 

\medskip
\needspace{6em}
\begin{tabular*}{7.1in}{@{}l@{\extracolsep\fill}r}
{\bf Stealth/vpNet} & Lead Engineer, 2002 - 2006\\
\end{tabular*}
%\textcolor{Periwinkle}{http://www.mak.com/products/stealth.php} \\
Designed and developed the MAK Stealth 6.0 series framework, GUI, and associated APIs (800+ classes, 12+ third party dependencies, and supporting features such as sound, joystick, effects, remote control, and custom model memory paging).  Between 2002 and 2005, Stealth sales doubled due to improvements in the tool. Aided project scheduling and management.

Designed and developed the StealthXR 1.0 series, an exaggerated reality visualizer for distributed simulation; Implemented features for the non-photorealistic rendering of entities and terrains as well as exaggerated non-perspective views to produce a god's eye overview of a battlefield.  In 2005, Stealth sales rose 34\% from 2004, 25\% of which were due to StealthXR sales. 

Worked with company partners in developing sister products and plug-ins; Helped resolve integration problems and setup demos, for example, with Joint Forces Command's (JFCOM) 2005 I/ITSEC Joint Virtual Training Special Event (JVTSE), which consisted of over 50+ participants, 1000+ entities, and 5Gb terrain playing together in a virtual environment
\medskip

\needspace{6em}
\begin{tabular*}{7.1in}{@{}l@{\extracolsep\fill}r}
{\bf VR-Forces Remote Control API} & 2002\\
\end{tabular*}
Designed and implemented version 1.0 of the network interface between a computer generated forces (CGF) simulator and front-end plan view display; Implemented TCP/IP,UDP socket communication and packet protocols; Designed API to abstract implementation details from users.
\medskip

\needspace{6em}
\begin{tabular*}{7.1in}{@{}l@{\extracolsep\fill}r}
{\bf RTI-Spy} & 2001 - 2002\\
\end{tabular*}
%\textcolor{Periwinkle}{http://www.mak.com/products/rti.php} \\
Designed and developed version 1.0 of the RTI Spy API and RTI Spy Console.  In 2002, RTI Spy sales helped contribute to nearly half of all RTI sales.
\medskip

\needspace{6em}
\begin{tabular*}{7.1in}{@{}l@{\extracolsep\fill}r}
{\bf VR-Link} & 2000 - 2002\\
\end{tabular*}
%\textcolor{Periwinkle}{http://www.mak.com/products/vrlink.php} \\
Implemented DIS/HLA protocols; Participated in SISO meetings during the design on the RPR FOM 1.0/2.0 standards.


